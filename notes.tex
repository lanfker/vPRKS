\documentclass[10pt]{article}
\usepackage[top=1in, bottom=1in, left=1.25in, right=1.25in]{geometry}
\usepackage[bookmarks=true]{hyperref}
\usepackage{setspace}
\usepackage{verbatim}
\usepackage{cite}
\usepackage{amsmath}

\begin{document}

\section{Question}
1. How to get $\Phi_i$ for Kalman filter????? (will work on this later)
2. How to design a message forwarding mechanism?

\section{To Do}
%check density of directions of nearby vehicles, allocate slots (by checking direction distribution, we can allocate slots distributedly)
%Maintain active slots in SignalMap. 
Maintain a vehicle active time (When the vehicle enters our simulation)


\section {Facts}
1. What information a packet has to contain:
\begin{itemize}
  \item tx power. double
  \item rx power. double
  \item travelling angle. Double
  \item current position (x,y). Two double
  \item begin sending slot in a frame. uint16\_t
  \item end sending slot in a frame  uint16\_t
\end{itemize}

\section {Thoughts}
1. We should not let vehicles send packets with abnormal high transmission power. This means when constructing signal map, we need message exchange.  Use prediction:

(a) Only store signal attenuation information if packets are succesfully received. 

(b) Use these records to predict path loss between nearby vehicles that are beyond transmission range.

(c) For predicted values, we should not store them in the observation matrix (vectors), only update them in the signal map.

We are actually using Control channel, and packets in control channel usually has much higher transmission power. As a result, we can create measured signal map beyond normal transmission range. (2014-04-01 Tue 11:44 AM)

Exclusion region is shared based a each link. A node has to compute the union of all its intended receiver's ER's. How to inform them?

How to decide if a node should stay in data channel? 1) If another explicitly tells the node to stay in data channel; 2) The node with the maximum priority within its union ER is a one-hop neighbor.


\section{How to run the protocol}
First, we need to create signal maps for our scheduling algorithm. When simulation starts, all nodes run CSMA/CA to create signal map for later use for 100 seconds.

For newly joined vehicles, they should first try to overhear nearby signal maps. Also, they should send out beacon messages in control channel to let others discover them.

Vehicles choose to stay in control channel if no vehicle asks them to stay in data channel. (Here, we need to make sure the message of informing other vehicles to stay in data channel be very reliable) 













\end{document}
